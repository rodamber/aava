\documentclass[12pt, a4paper]{article}

\usepackage[T1]{fontenc}
\usepackage[utf8]{inputenc}
\usepackage{authblk}
\usepackage{enumitem}
\usepackage{graphicx}
\usepackage{tikz}
\usepackage{pgfplots}

\usetikzlibrary{matrix}
\pgfplotsset{compat=1.13}
\usepgfplotslibrary{groupplots}
\usetikzlibrary{pgfplots.groupplots}

\title{\textbf{Advanced Algorithms Project Report}}
\author{Rodrigo Bernardo \\ ist178942}
\affil{Instituto Superior Técnico}

\date{\today}

\begin{document}
\maketitle

\section{Text Searching}
\subsection{Overview}
In this section we discuss the implementation and profiling of the naive,
Knuth-Morris-Pratt and Boyer-Moore algorithms for string matching. The
implementation for the Knuth-Morris-Pratt algorithm was taken from the
third edition of the book \textit{Introduction to Algorithms}, while the
implementation for the Boyer-Moore algorithm was taken from \textit{Algorithms
on Strings, Trees, and Sequences}.

\subsection{Tests and Expectations}
In this project the naive algorithm serves as a baseline for testing the other
two algorithms, which we expect to be much faster in all situations.
\paragraph{}
We profile time and heap usage, as well as the number of comparisons each
algorithm performs. We test the algorithms on randomly generated text and
pattern strings consisting of 'A', 'T', 'C' and 'G' characters only.
\paragraph{}
We expect the Knuth-Morris-Pratt and Boyer-Moore to run on $O(n+m)$ both on time
and space, where $n$ is the length of the text and $m$ is the length of pattern.
The complexity of the naive algorithm is also $O(n+m)$ in the average case, but
it decays to $O(nm)$ in the worst case.

\subsection{Results}

%------------------------------------------------------------------------------%
% Time Profiling
\begin{tikzpicture}
  \pgfplotsset{every axis legend/.append style={
      at={(0,1)},
      anchor=north west}}
  \begin{groupplot}[
    group style={
      group name=Time Profiling,
      group size=1 by 3,
      %
      horizontal sep=1.5cm,
      vertical sep=1.5cm,
      %
      xlabels at=edge bottom,
      ylabels at=edge left,
    },
    footnotesize,
    enlargelimits=false,
    %
    width=0.8\textwidth,
    height=7cm,
    %
    xlabel=text size $n$ / characters,
    ylabel=time $t$ / ms,
    %
    xtick={1e5,2e5,3e5,4e5,5e5,6e5,7e5,8e5,9e5,1e6},
    %
    xmajorgrids=true,
    ymajorgrids=true,
    grid style=dashed,
    %
    legend entries={$m=16$,$m=64$,$m=256$},
    ]
    %
    \nextgroupplot[title=Naive]
      \addplot+[only marks, mark=square*] table [x=n,y=time] {profile/data/naive16.dat};
      \addplot+[only marks, mark=triangle*] table [x=n,y=time] {profile/data/naive64.dat};
      \addplot+[only marks, mark=o] table [x=n,y=time] {profile/data/naive256.dat};
    \nextgroupplot[title=Knuth-Morris-Pratt]
      \addplot+[only marks, mark=square*] table [x=n,y=time] {profile/data/kmp16.dat};
      \addplot+[only marks, mark=triangle*] table [x=n,y=time] {profile/data/kmp64.dat};
      \addplot+[only marks, mark=o] table [x=n,y=time] {profile/data/kmp256.dat};
    \nextgroupplot[title=Boyer-Moore]
      \addplot+[only marks, mark=square*] table [x=n,y=time] {profile/data/bm16.dat};
      \addplot+[only marks, mark=triangle*] table [x=n,y=time] {profile/data/bm64.dat};
      \addplot+[only marks, mark=o] table [x=n,y=time] {profile/data/bm256.dat};
  \end{groupplot}
  \node[text width=6cm,align=center,anchor=north] at ([yshift=+15mm]Time Profiling c1r1.north) {Time Profiling};
\end{tikzpicture}

%------------------------------------------------------------------------------%
% Heap Profiling
\begin{tikzpicture}
  \pgfplotsset{every axis legend/.append style={
      at={(0,1)},
      anchor=north west}}
  \begin{groupplot}[
    group style={
      group name=Heap Profiling,
      group size=1 by 3,
      %
      horizontal sep=1.5cm,
      vertical sep=1.5cm,
      %
      xlabels at=edge bottom,
      ylabels at=edge left,
    },
    footnotesize,
    enlargelimits=false,
    %
    width=0.8\textwidth,
    height=7cm,
    %
    xlabel=text size $n$ / characters,
    ylabel=heap / bytes,
    %
    xtick={1e5,2e5,3e5,4e5,5e5,6e5,7e5,8e5,9e5,1e6},
    %
    xmajorgrids=true,
    ymajorgrids=true,
    grid style=dashed,
    %
    legend entries={$m=16$,$m=64$,$m=256$},
    ]
    %
    \nextgroupplot[title=Naive]
      \addplot+[only marks, mark=square*] table [x=n,y=heap] {profile/data/naive16.dat};
      \addplot+[only marks, mark=triangle*] table [x=n,y=heap] {profile/data/naive64.dat};
      \addplot+[only marks, mark=o] table [x=n,y=heap] {profile/data/naive256.dat};
    \nextgroupplot[title=Knuth-Morris-Pratt]
      \addplot+[only marks, mark=square*] table [x=n,y=heap] {profile/data/kmp16.dat};
      \addplot+[only marks, mark=triangle*] table [x=n,y=heap] {profile/data/kmp64.dat};
      \addplot+[only marks, mark=o] table [x=n,y=heap] {profile/data/kmp256.dat};
    \nextgroupplot[title=Boyer-Moore]
      \addplot+[only marks, mark=square*] table [x=n,y=heap] {profile/data/bm16.dat};
      \addplot+[only marks, mark=triangle*] table [x=n,y=heap] {profile/data/bm64.dat};
      \addplot+[only marks, mark=o] table [x=n,y=heap] {profile/data/bm256.dat};
  \end{groupplot}
  \node[text width=6cm,align=center,anchor=north] at ([yshift=+15mm]Heap Profiling c1r1.north) {Heap Profiling};
\end{tikzpicture}

%------------------------------------------------------------------------------%
% Comparisons Count
\begin{tikzpicture}
  \pgfplotsset{every axis legend/.append style={
      at={(0,1)},
      anchor=north west}}
  \begin{groupplot}[
    group style={
      group name=Comparisons Count,
      group size=1 by 3,
      %
      horizontal sep=1.5cm,
      vertical sep=1.5cm,
      %
      xlabels at=edge bottom,
      ylabels at=edge left,
    },
    footnotesize,
    enlargelimits=false,
    %
    max space between ticks=25pt,
    %
    width=0.8\textwidth,
    height=7cm,
    %
    xlabel=text size $n$ / characters,
    ylabel=comparisons \#,
    %
    xtick={1e5,2e5,3e5,4e5,5e5,6e5,7e5,8e5,9e5,1e6},
    %
    xmajorgrids=true,
    ymajorgrids=true,
    grid style=dashed,
    %
    legend entries={$m=16$,$m=64$,$m=256$},
    ]
    %
    \nextgroupplot[title=Naive]
      \addplot+[only marks, mark=square*] table [x=n,y=comparisons] {profile/data/naive16.dat};
      \addplot+[only marks, mark=triangle*] table [x=n,y=comparisons] {profile/data/naive64.dat};
      \addplot+[only marks, mark=o] table [x=n,y=comparisons] {profile/data/naive256.dat};
    \nextgroupplot[title=Knuth-Morris-Pratt]
      \addplot+[only marks, mark=square*] table [x=n,y=comparisons] {profile/data/kmp16.dat};
      \addplot+[only marks, mark=triangle*] table [x=n,y=comparisons] {profile/data/kmp64.dat};
      \addplot+[only marks, mark=o] table [x=n,y=comparisons] {profile/data/kmp256.dat};
    \nextgroupplot[title=Boyer-Moore]
      \addplot+[only marks, mark=square*] table [x=n,y=comparisons] {profile/data/bm16.dat};
      \addplot+[only marks, mark=triangle*] table [x=n,y=comparisons] {profile/data/bm64.dat};
      \addplot+[only marks, mark=o] table [x=n,y=comparisons] {profile/data/bm256.dat};
  \end{groupplot}
  \node[text width=6cm,align=center,anchor=north] at ([yshift=+15mm]Comparisons Count c1r1.north) {Comparisons Count};
\end{tikzpicture}

%------------------------------------------------------------------------------%
% Answers
\begin{tikzpicture}
  \pgfplotsset{every axis legend/.append style={
      at={(0,1)},
      anchor=north west}}
  \begin{groupplot}[
    group style={
      group name=Corner Cases,
      group size=1 by 3,
      %
      horizontal sep=1.5cm,
      vertical sep=1.5cm,
      %
      xlabels at=edge bottom,
      ylabels at=edge left,
    },
    footnotesize,
    enlargelimits=false,
    %
    width=0.8\textwidth,
    height=7cm,
    %
    xlabel=text size $n$ / characters,
    ylabel=comparisons \#,
    %
    xtick={1e5,2e5,3e5,4e5,5e5,6e5,7e5,8e5,9e5,1e6},
    %
    xmajorgrids=true,
    ymajorgrids=true,
    grid style=dashed,
    %
    legend entries={$m=16$,$m=64$,$m=256$},
    ]
    %
    \nextgroupplot[title=Naive Bad Case]
      \addplot+[only marks, mark=square*] table [x=n,y=comparisons] {profile/data/naive-bad16.dat};
      \addplot+[only marks, mark=triangle*] table [x=n,y=comparisons] {profile/data/naive-bad64.dat};
      \addplot+[only marks, mark=o] table [x=n,y=comparisons] {profile/data/naive-bad256.dat};
    \nextgroupplot[title=Boyer-Moore Sublinear]
      \addplot+[only marks, mark=square*] table [x=n,y=comparisons] {profile/data/bm-sublinear16.dat};
      \addplot+[only marks, mark=triangle*] table [x=n,y=comparisons] {profile/data/bm-sublinear64.dat};
      \addplot+[only marks, mark=o] table [x=n,y=comparisons] {profile/data/bm-sublinear256.dat};
  \end{groupplot}
  \node[text width=6cm,align=center,anchor=north] at ([yshift=+15mm]Comparisons Count c1r1.north) {Answers to Questions};
\end{tikzpicture}

\subsection{Results Analysis and Conclusions}
\subsubsection{Time and Heap Profiling}
We can check that all the algorithms run in linear time and space, and that
Knuth-Morris-Pratt and Boyer-Moore have much better performance than naive. The
naive and Knuth-Morris-Pratt running times don't seem to be much affected by the
increase in size of the pattern. Boyer-Moore actually runs faster and in less
space with bigger patterns.
\subsubsection{Answers to the Questions in the Project Specification}
The naive algorithm worst-case happens when the text has many similar substrings
to the pattern. In the penultimate plot we can see the result of an experience
were we applied the naive algorithm to texts and patterns with these
caracteristics. We see, as expected, that the algorithm makes many more
comparisons the bigger the pattern.

In the last plot we find the answer to why the Boyer-Moore is claimed to be
sublinear. We applied this algorithm against texts and patterns where the
matching process failed quickly. By use of the bad character, the algorithm
manages to skip a lot of characters, thus managing to perform fewer comparisons.
By abusing this rule the algorithm may reach the end of the text without making
$O(n+m)$ comparisons, as we can see in the plot.

\section{Connectivity in Forests}
\subsection{Overview}
In this problem we aim to maintain the connectivity information of a forest. For
that we use an implementation of Link-Cut trees with Splay Trees, based on the
paper \textit{Self-adjusting binary trees}, by Sleator and Tarjan. 
\subsection{Tests and Expectations}
We test the implementation time and space performance. For each $n$ (the number
of nodes), we start by building a linear tree. A cut then follows, cutting the
tree in two halves. Then we apply 2000 random operations (a link, a cut or
a connected query), and measure the time those operations take.

We're expecting linear increases in space usage with the increase of the number
of nodes, as the algorithm only does one explicit memory allocation per node,
and doesn't use recursion. Also, we're expecting the tree's operations to run
all in $log(n)$ amortized time.

\subsection{Results}

%------------------------------------------------------------------------------%
% Link-Cut
\begin{tikzpicture}
  \pgfplotsset{every axis legend/.append style={
      at={(0,1)},
      anchor=north west}}
  \begin{groupplot}[
    group style={
      group name=Link-Cut Tree,
      group size=1 by 2,
      %
      horizontal sep=1cm,
      vertical sep=1.5cm,
      %
      xlabels at=edge bottom,
      ylabels at=edge left,
    },
    footnotesize,
    enlargelimits=false,
    %
    width=0.8\textwidth,
    height=7cm,
    %
    xlabel=Total number of nodes $n$,
    %
    % xtick={1e3,1e4,1e5,1e6,1e7},
    %
    xmajorgrids=true,
    ymajorgrids=true,
    grid style=dashed,
    ]
    %
    \nextgroupplot[ylabel=time $t$ / ms,]
      \addplot+[only marks, mark=*] table [x=n,y=time] {profile/data/link-cut-times.dat};
    \nextgroupplot[ylabel=heap / bytes, scaled y ticks=base 10:-6,]
      \addplot+[only marks, mark=*,] table [x=n,y=heap] {profile/data/link-cut.dat};
  \end{groupplot}
  \node[text width=6cm,align=center,anchor=north] at ([yshift=+15mm]Link-Cut Tree c1r1.north) {Link-Cut Tree};
\end{tikzpicture}

\subsection{Results Analysis}
The time plot shows a curve that resembles a logarithm. While we cannot assert
that each operation runs, by itself, in $log(n)$, we can assert that in an
amortized sense.

The space usage is completely linear on the number of nodes, as expected.

\end{document}
